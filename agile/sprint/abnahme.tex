% rubber: set program xelatex
\documentclass[alpha, signed]{tile-sprint} % Option path/to/src may be required!
% Document
\title{Sprint x}
\subtitle{Abnahme}							% Optional
\author{Max Mustermann, Minni Musterfrau}
% Sprint
\sprint{Sprint x}							% Sprint name
\sprintstart{dd. Month yyyy}				% Date the sprint started
\sprintend{dd. Month yyyy}					% Date the sprint ends
% Project
\release{Alpha}								% Current product release
% TILE Article
\creationname{Max Mustermann} 				% Creator of this document 			- Optional
\creationdate{dd.mm.yyyy} 					% Date this document was created 	- Optional
\validationname{Max Mustermann}				% Validator of this document 		- Optional
\validationdate{dd.mm.yyyy} 				% Date this document was validated	- Optional
\approvedname{Max Mustermann} 				% Approver of this document			- Optional
\approveddate{dd.mm.yyyy}					% Date this document was approved	- Optional
\version{x.x.x}								% Version of this document
\filepath{presets/sprint/tile-sprint-x-abnahme.pdf}	% Git path to this document

\begin{document}
\begin{changelog}
    \addversion{x.x.x}{dd.mm.yyyy}{MUSM}{Lorem ipsum dolor sit amet. Lorem ipsum dolor sit amet.}
    \addversion{x.x.x}{dd.mm.yyyy}{MUSM}{Lorem ipsum dolor sit amet. Lorem ipsum dolor sit amet.}
\end{changelog}

\newpage
\section{Ziel}
\section{Sprint Backlog}
\subsection{Abgeschlossen}
\subsubsection{Übernommen}
\begin{backlog}
	\addbacklogitem{xx}{Mustertask 1}{Type}{x.y}{x.y}{x.y}{Bearbeiter}{Status}
	\addbacklogitem{xx}{Mustertask 2}{Type}{x.y}{x.y}{x.y}{Bearbeiter}{Status}
	\addbacklogsum{x.y}{x.y}{x.y}
\end{backlog}
\subsubsection{Hinzugefügt}
\begin{backlog}
	\addbacklogitem{xx}{Mustertask 3}{Type}{x.y}{x.y}{x.y}{Bearbeiter}{Status}
	\addbacklogitem{xx}{Mustertask 4}{Type}{x.y}{x.y}{x.y}{Bearbeiter}{Status}
	\addbacklogsum{x.y}{x.y}{x.y}
\end{backlog}
\subsection{Verschoben}
\subsubsection{Bearbeitet}
\begin{backlog}
	\addbacklogitem{xx}{Mustertask 5}{Type}{x.y}{x.y}{x.y}{Bearbeiter}{Status}
	\addbacklogitem{xx}{Mustertask 6}{Type}{x.y}{x.y}{x.y}{Bearbeiter}{Status}
	\addbacklogsum{x.y}{x.y}{x.y}
\end{backlog}

\newpage
\section{Kommentare}
\makecommenttable{20}
\section{Abnahme}
Der oben referenzierte Sprint, {\makeatletter\@sprint\makeatother}, gilt mit dem {\makeatletter\@sprintend\makeatother} als beendet und wird hiermit durch den Product Owner abgenommen. \vspace{0.5em} \\
Offene Elemente des Sprint Backlogs werden damit in den Product Backlog verschoben.
\\ \vspace{0.5em}
\begin{center}
	\rule{0.25 \textwidth}{\headrulewidth} ~\\ \vspace{0.5em}
	Markus Reichl \\
	\footnotesize{(Product Owner)}
\end{center}

\end{document}
