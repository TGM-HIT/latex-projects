\section{Vertragsgegenstand}
Es werden die Kriterien genannt, unter denen das Produkt vom Auftragnehmer den Auftraggeber übergeben wird. Es befasst den Lieferumfang, die Produktleistung und die Produktbezogene Leistungen. 
\subsection{Lieferumfang}
Nach Abschluss des Projekts erhält der Auftraggeber einen USB-Stick auf dem sich die App und Desktop-Version befindet. Außerdem beinhaltet dieser USB-Stick die Webseite welche dann hochgeladen wird.
\subsection{Produktleistungen}
Es werden die Leistungen aufgezählt, welche vom fertigen Produkt erfüllt werden. Diese sollen vom Auftragnehmer erfüllt werden, es sei den diese sind von der Umsetzung her bewiesen nicht möglich.
\begin{indentE}\mbox{}
	\paragraph{/LL1010/ Produktgröße}\mbox{}\\
	Die Gesamtgröße des Endprodukt soll für ein System nicht größer sein als 500MB. Das heißt, das weder die App noch Desktopversion eine Größe von 500MB überschreiten darf.
	\paragraph{/LL1020/ Datenübertragungsrate}\mbox{}\\
	Es soll mit dem Produkt möglich sein mit einer Datentransferrate von bis zu 25 Mbit/s mit Bluetooth Version 5.0 Daten zu übertragen, diese kann bei jüngeren Bluetooth Versionen geringer sein.
	\paragraph{/LL1030/ Dateigröße}\mbox{}\\
	Das Produkt sollen Daten bis zu einer Größen von 1GB problemlos übertragen werden können.
	\paragraph{/LF2110/ Windows kompatibel}\mbox{}\\
	Das System muss für Windows 10 kompatibel sein, und auf diesen Betriebssystem mit all seinen Funktionen funktionieren.
	\paragraph{/LF2120/ Debian kompatibel}\mbox{}\\
	Das System sollte für Debian 9 kompatibel sein, und auf diesen Betriebssystem mit all seinen Funktionen funktionieren.
	\paragraph{/LF2130/ macOS kompatibel}\mbox{}\\
	Das System kann für macOs kompatibel sein, und auf diesen Betriebssystem mit all seinen Funktionen funktionieren.
	\paragraph{/LF2210/ Android kompatibel}\mbox{}\\
	Das System muss für Android 8.0 kompatibel sein, und auf diesen Betriebssystem mit all seinen Funktionen funktionieren.
	\paragraph{/LF2220/ iOS kompatibel}\mbox{}\\
	Das System kann für iOS kompatibel sein, und auf diesen Betriebssystem mit all seinen Funktionen funktionieren.
\end{indentE}
\subsection{Produktbezogene Leistungen}
Der Auftragnehmer verpflichtet sich nicht dazu das Projekt nach Lieferung zu warten, zu schulen oder zu betreiben. Falls in folge eines halben Jahres auffällt das Funktionen fehlen oder nicht funktionieren ist der Auftragnehmer verpflichtet dazu diese nachträglich beizusetzen.