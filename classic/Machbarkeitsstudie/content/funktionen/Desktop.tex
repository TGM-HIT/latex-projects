\subsection{Desktop-Version}
Es werden die Funktionen beschrieben, welche die Desktop-Version des Produktes erfüllen soll. Diese werden grundsätzliche in Frontend und Backend unterteilt.
\subsubsection{Backend erstellen}
Das Backend umfasst alle Funktionen, welche vom Benutzer nicht gesehen werden können. Diese bearbeitet den Datentransfer, Datenaufbereitung und die Datensicherheit.
\paragraph{Aktivitätsdiagramm Backend}\mbox{}\\
\begin{figure}[H]
	\centering
	\includegraphics[width= 0.9\linewidth]{diagramms/activity/Backend.png}
	\caption{Aktivitätsdiagramm Backend}
\end{figure}
\newpage
\begin{indentE}\mbox{}
	\paragraph{/LF1110/ Verbindung aufbauen}\mbox{}\\
	Es wird die Verbindung mit dem ausgewählten Empfänger aufgebaut. Dieser Verbindungsaufbau besteht aus einem verschlüsselten Handshake, mit welchem die Systemdaten des Empfängers und Senders ausgetauscht werden. Wenn Empfänger und Sender den selben Verschlüsselungsschlüssel gewählt haben, ist der Handshake erfolgreich und der Transfer kann beginnen.
	\UseCase{
		{Desktop Backend}
		{/LF1110/ Verbindung aufbauen}
		{Backend}
		{Es wird die Verbindung mit dem Empfänger als Sender aufgebaut.}
		{Es muss eine Verbindung zwischen Empfänger und Sender für den Datentransfer herrschen}
		{Der Empfänger kann sich mit einem Empfänger der Wahl verbinden.}
		{Empfänger, Sender}
		{Sender, Namen }
		{Der Sender kann sich nicht mit dem Empfänger verbinden.}
		{Der Sender kann sich mit dem Empfänger verbinden.}
		{hoch}
		{mittel}
		{Must Have}
	}
	\paragraph{/LF1120/ Daten transferieren}\mbox{}\\
	Die Daten werden in kleinen Paketen versandt, um die Datensicherheit zu garantieren. Dabei haben diese eine einheitliche Größe und es werden unterschiedliche viele je nach Datengröße versandt. Dabei sollen der Empfänger und der Sender um den Fortschritt der Übertragung aufgeklärt werden.
	\UseCase{
		{Desktop Backend}
		{/LF1120/ Daten transferieren}
		{Backend}
		{Die Daten sollen vom Sender zum Empfänger transferiert werden. Diese soll dabei möglichst verlustlos gesendet werden.}
		{Der Daten müssen an den Empfänger gebracht werden.}
		{Die Daten des Sender kommen beim Empfänger an.}
		{Empfänger, Sender}
		{Datei(Name, Größe, Inhalt), Fortschritt(aus index und Anzahl Pakete), Sender, Empfänger}
		{Die Daten können dem Empfänger nicht vom Sender geschickt werden.}
		{Die Daten werden vom Sender zum Empfänger verlustlos gesendet.}
		{hoch}
		{gering}
		{Must Have}
	}
	\paragraph{/LF1130/ Daten komprimieren}\mbox{}\\
	Die Daten werden vor dem Versand verlustlos komprimiert, um die Datenmenge zu verkleinern. Die dabei gewählt Komprimierungsart, darf vom Auftragnehmer ausgewählt werden.
	\UseCase{
		{Desktop Backend}
		{/LF1130/ Daten komprimieren}
		{Backend}
		{Die Daten sollen vor dem Transfer verlustlos komprimiert werden, um die Übertragungsdauer zu verringern}
		{Der Datentransfer soll kürzer dauern.}
		{Der Datentransfer dauert in den meisten Fällen kürzer.}
		{Sender, Empfänger}
		{Daten}
		{Die Datenübertragung dauert länger.}
		{Die Datenübertragung dauert kürzer.}
		{mittel}
		{mittel}
		{Should Have}
	}
	\paragraph{/LF1140/ Datentransfer verschlüsseln}\mbox{}\\
	Der Datentransfer wird mit einem Verschlüsselungsschlüssel nach dem AES256 Standard verschlüsselt, um die Datensicherheit zu erhöhen. Diese wird dann vom Empfänger wieder entschlüsselt. Die dafür benutzte Verschlüsselungsart, kann vom Auftraggeber ausgewählt werden.
	\UseCase{
		{Desktop Backend}
		{/LF1140/ Datentransfer verschlüsseln}
		{Backend}
		{Der Datentransfer soll für eine höhere Sicherheit durch einen Schlüssel verschlüsselt werden.}
		{Der Datentransfer soll verschlüsselt stattfinden, damit die originalen Daten nicht von Dritten ausgelesen werden können.}
		{Der Datentransfer erfüllt den Standard AES256.}
		{Empfänger, Sender, Dritte}
		{Daten, Schlüssel}
		{Die originalen Daten können von Dritten ausgelesen werden}
		{Die originalen Daten sind während dem Transfer verschlüsselt}
		{mittel}
		{mittel}
		{Must Have}
	}
	\paragraph{/LF1150/ Benutzereinstellungen integrieren}\mbox{}\\
	Die vom Benutzer ausgewählten Benutzereinstellungen müssen in das Backend integrierte werden, um den Nutzen aus diesen Werten zu ziehen.
	\UseCase{
		{Desktop Backend}
		{/LF1150/ Benutzereinstellungen integrieren}
		{Backend}
		{Die Benutzereinstellungen sollen integriert werden.}
		{Die ausgewählten Benutzereinstellungen sollen in das Backend integriert werden.}
		{Die Einstellungen des Benutzer haben eine Auswirkung auf das Backend.}
		{Sender, Empfänger}
		{gewünschte Systemname, Verschlüsselung (Ja/Nein), Standardschlüssel, Speicherort}
		{Der Benutzer muss die meisten Daten bei jeder Übertragung neu einstellen, wodurch diese länger dauert}
		{Der Übertragung ist anpassbar, wodurch diese verschnellert werden kann. }
		{mittel}
		{gering}
		{Should Have}
	}
\end{indentE}
\subsubsection{Frontend erstellen}
Zu dem Frontend des Produktes gehören alle Elemente, auf welche der Benutzer Einsicht hat und mit denen er interagieren kann. Diese sind besonders für die Benutzbarkeit des Produkt wichtig, da diese die wichtigste Produktqualität ist.
\paragraph{Aktivitätsdiagramm Frontend}\mbox{}\\
\begin{figure}[H]
	\centering
	\includegraphics[width= 0.9\linewidth]{diagramms/activity/Frontend.png}
	\caption{Aktivitätsdiagramm Frontend}
\end{figure}
\newpage
\begin{indentE}\mbox{}
	\paragraph{/LF1210/ Datei auswählen}\mbox{}\\
	Der Benutzer wählt eine Datei aus, welche er verschicken möchte. Hierbei soll eine voreilige Kalkulation durchgeführt werden, welche die Dauer der Übertragung schätzt.
	\UseCase{
		{Desktop Frontend}
		{/LF1210/ Datei auswählen}
		{Frontend}
		{Ein wird die zu versendende Datei ausgewählt.}
		{Es muss die Datei, welche versendet werden soll, ausgewählt werden}
		{Es kann die Datei gesendet werden}
		{Sender}
		{Datei(Name, Größe, Inhalt)}
		{Es kann keine Datei ausgewählt werden}
		{Es kann eine zu versendende Datei ausgewählt werden}
		{hoch}
		{mittel}
		{Must Have}
	}
	\paragraph{/LF1220/ Empfänger auswählen}\mbox{}\\
	Es werden dem Benutzer die möglichen Empfänger angezeigt und ausgwählt oder er kann den Namen des Empfängers eingeben, um den Empfänger zu bestimmen. Dieser muss die Verbindung bestätigen.
	\UseCase{
		{Desktop Frontend}
		{/LF1220/ Empfänger auswählen}
		{Frontend}
		{Der Sender wählt anhand des Namens den Empfänger aus, mit dem der Datentransfer stattfinden soll.}
		{Es muss der Empfänger ausgewählt werden.}
		{Die Verbindung kann aufgebaut werden}
		{Sender, Empfänger}
		{Sendername, Empfängername}
		{Es kann kein Empfänger ausgewählt werden.}
		{Es kann anhand des Namens der Empfänger der Datei ausgewählt werden.}
		{hoch}
		{mittel}
		{Must Have}
	}
	\paragraph{/LF1230/ Verschlüsselungsschlüssel eingeben}\mbox{}\\
	Wenn kein Standard-Schlüssel ausgewählt ist, muss der Sender diesen bevor dem Verbindungsaufbau eingeben. Dieser muss dann dem Empfänger mitgeteilt werden und dieser muss den selben auch eingeben, um den Datentransfer zu ermöglichen.
	\UseCase{
		{Desktop Frontend}
		{/LF1230/ Verschlüsselungsschlüssel eingeben}
		{Frontend}
		{Es müssen Empfänger und Sender einen Schlüssel eingeben, die übereinstimmen, um den Datentransfer zu beginnen.}
		{Der Datentransfer muss mit einem Schlüssel verschlüsselt werden.}
		{Der Datentransfer kann verschlüsselt werden}
		{Sender, Empfänger}
		{Schlüssel}
		{Die Daten können nicht mit einem Schlüssel verschlüsselt werden}
		{Die Daten können mit einem frei wählbaren Schlüssel verschlüsselt werden}
		{hoch}
		{mittel}
		{Must Have}
	}
	\paragraph{/LF1240/ Datei speichern}\mbox{}\\
	Der Empfänger soll nach dem empfangen der Datei auswählen können, ob diese im Standard-Speicherort gespeichert wird, oder in einem besonderen. Wenn der Besondere ausgewählt worden ist, soll ihm ein Benutzerinterface gezeigt werden, indem er den Speicherort auswählen.
	\UseCase{
		{Desktop Frontend}
		{/LF1240/ Datei speichern}
		{Frontend}
		{Der Empfänger soll den Speicherort des bekommenen Datei auswählen können.}
		{Die Datei muss irgendwo gespeichert werden}
		{Der Speicherort der Datei kann ausgewählt werden}
		{Empfänger}
		{Datei(Name, Größe, Inhalt), Speicherort}
		{Die Datei wird entweder immer am selben Ort, oder nicht gespeichert}
		{Der Speicherort kann frei gewählt werden}
		{mittel}
		{gering}
		{Must Have}
	}
	\paragraph{/LF1250/ Benutzereinstellungen hinzufügen}\mbox{}\\
	Der Benutzer kann einige Fix-Einstellungen auf seinem Produkt tätigen. Darunter zählt die Möglichkeit, den Namen zu ändern, welcher den restlichen Nutzern angezeigt wird, die Verschlüsselung zu aktivieren oder deaktivieren, einen Standardschlüssel auszuwählen und einen Standard-Speicherort auszuwählen. Dabei sollen die eingegebenen Daten auf ihre Korrektheit überprüft werden.
	\UseCase{
		{Desktop Frontend}
		{/LF1250/ Benutzereinstellungen hinzufügen}
		{Frontend}
		{Der Benutzer kann verschiedenen Einstellungen tätigen, mit welche er die Datenübertragung beeinflussen kann.}
		{Die Einstellungen müssen auf einem Interface getätigt werden}
		{Die Einstellungen können getätigt werden}
		{Empfänger, Sender}
		{gewünschter Systemname, Verschlüsselung(Ja/Nein), Standardschlüssel, Standard-Speicherort}
		{Das Produkt kann nicht angepasst werden.}
		{Es können Einstellungen getroffen werden, welche das Produkt beeinflussen.}
		{mittel}
		{mittel}
		{Should Have}
	}
	\paragraph{/LF1260/ Desktopinterface designen}\mbox{}\\
	Es muss ein Interface für den Desktop entworfen werden. Dazu gehören Mockups und Prototypen. Diese müssen vor deren Implementierung vom Auftraggeber bestätigt werden.
	\UseCase{
		{Desktop Frontend}
		{/LF1260/ Desktopinterface designen}
		{Frontend}
		{Es müssen Mockups für die Desktopinterfaces erstellt werden. Diese diene als Vorlage für die Interfaces.}
		{Für einen schnellere und besser Implementierung sollen Mockups als Vorlage dienen.}
		{Die Implementierung ist einfacher, schneller und besser.}
		{Auftraggeber, Auftragnehmer}
		{}
		{Das Interface wird ohne Vorlage erstellt.}
		{Das Interface kann nach einer vom Auftraggeber bestätigten Vorlage erstellt werden.}
		{mittel}
		{gering}
		{Should Have}
	}
	\paragraph{/LF1270/ Desktopinterace implementieren}\mbox{}\\
	Das Desktopinterface muss nach der Bestätigung des Auftraggeber implementiert werden. 
	\UseCase{
		{Desktop Frontend}
		{/LF1270/}
		{Frontend}
		{Das Desktopinterface muss durch die Mockups realisiert werden.}
		{Der Benutzer hat keine Möglichkeit mit dem System zu integrieren}
		{Der Benutzer kann mit dem System interagieren und dadurch dessen Funktionen benutzen.}
		{Benutzer}
		{sämtliche Nutzereingaben}
		{Der Benutzer kann mit dem System nicht interagieren}
		{Der Benutzer ist in der Lage mit dem System zu interagieren}
		{hoch}
		{hoch}
		{Must-Have}
	}
	\paragraph{/LF1280/ Frontend und Backend verbinden}\mbox{}\\
	Es muss eine Verbindung zwischen dem Frontend und Backend erstellt werden, welche die Aufforderungen verarbeitet und zum Backend weiterleitet. Diese Verbindung, soll dem Benutzer entweder mitteilen, wie lange die Verarbeitung noch dauert oder bei kürzeren das wiederholte Aufrufen von Aufforderungen verhindern.
	\UseCase{
		{Desktop Frontend}
		{/LF1280/ Frontend und Backend verbinden}
		{Frontend}
		{Das Frontend muss mit dem Backend verbunden werden, um die Nutzereingaben an das Backend weiterzuleiten. Außerdem soll es Daten ans Frontend schicken, welche über momentane Prozess aufklärt.}
		{Die Daten müssen vom Frontend ans Backend und umgekehrt geschickt werden}
		{Das Frontend kann mit dem Backend und umgekehrt kommunizieren.}
		{Benutzer}
		{sämtliche Nutzereingaben und verwertete Daten}
		{Die Nutzerdaten können nicht an das Backend und die verwerteten Daten nicht ans Frontend geschickt werden.}
		{Daten können zwischen Frontend und Backend ausgetauscht werden.}
		{hoch}
		{mittel}
		{Must-Have}
	}
\end{indentE}