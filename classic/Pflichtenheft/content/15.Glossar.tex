\section{Glossar}
Das \textbf{Backend} beschreibt die Schnittstelle zwischen einem Rechengerät und einer Software. Sie enthält die Funktionen, meist in einer Programmiersprache umgesetzt. Backend und Frontend sind zu unterscheiden.
\\
\\
\textbf{Windows 10} ist das aktuellste Betriebssystem der Microsoft Windows NT Serie. Es stellt ein System für ein Rechengerät dar.
\\
\\
\textbf{Android 8.0} oder auch 'Oreo' genannt. Ist die am häufigst genutzte Google Android Version, welche noch aktuell ist. Sie bietet ein Betriebssystem für Smartphones, jedoch nicht für Apple iPhones. Android ist meist OEM und kostenfrei.
\\
\\
\textbf{Debian 9} ist eine sehr beliebte Linux-Distribution, die noch oft im Einsatz von kleineren Firmen und Privatanwendern ist. Es ist ein kostenfreies Betriebssystem.
\\
\\
\textbf{macOS} ist das einzige Betriebssystem für Rechensysteme von Apple und wird nur für die Eigenmarken verwendet.
\\
\\
\textbf{iOS} ist das einzige Betriebssystem für Smartphones von Apple und wird ebenso nur für die Eigenmarken verwendet.
\\
\\
Mit dem \textbf{Frontend} wird die Schnittstelle zwischen der Software und dem Anwender beschrieben. Sie wird auch oft als grafische Darstellung einer Software bezeichnet.
\\
\\
Mit dem \textbf{Desktopinterface} ist das Frontend der Desktopapplikation gemeint.
\\
\\
Mit dem \textbf{Appinterface} ist das Frontend der App gemeint.
\\
\\
Der \textbf{Play Store} ist das Application-Prividing-System von Google auf dem Android-Betriebssystem.
\\
\\
Der \textbf{App Store} ist das Application-Providing-System von Apple auf dem macOS- und iOS-Betriebssystem.
\\
\\
\textbf{BlueTooth} beschreibt die Übertragungsschnittstelle, welche nur auf einem kleinen Bereich wirksam ist. Sie entspricht dem Standard IEEE 802.15.1.