\newacronym{ac-syt}{SYT}{Systemtechnik}
\newglossaryentry{syt}{
	name={Systemtechnik},
	description={\enquote{Als Systemtechnik bezeichnet man verschiedene Aufbau- und Verbindungstechniken, aber auch eine Fachrichtung der Ingenieurwissenschaften. Er bedeutet in der Unterscheidung zu den Mikrotechnologien die Verbindung verschiedener einzelner Module eines Systems und deren Konzeption.} \cite{wiki:syt}}
}
\newglossaryentry{dic}{
	name={Dictionary},
	description={\enquote{Als Dictionary bezeichnet in Programmiersprachen meist eine Schlüssel zu Wert Zuweisung. Beispielsweise kann über den Schlüssel "ID" auf den Wert "Schülername" zugegriffen werden.}}
}
\newglossaryentry{timst}{
	name={Timestamp},
	description={\enquote{Ein Zeitstempel (englisch timestamp) wird benutzt, um einem Ereignis einen eindeutigen Zeitpunkt zuzuordnen.} \cite{https://de.wikipedia.org/wiki/Zeitstempel}}
}
\newglossaryentry{lbvt}{
	name={LBVT},
	description={\enquote{Die Abkürzung von Lernbüroverwaltungstool.}}
}

\newglossaryentry{lba}{
	name={Lernbüroapplikation},
	description={\enquote{Bezeichnet das derzeitige System, dass für die manuelle Anwesenheitskontrolle und Administration des Lernbüros verantwortlich ist.}}
}
\newglossaryentry{git-repo}{
	name={Git Repository},
	description={\enquote{Ein Versionierungssystem welches gut dazu geeignet ist um Programmcode zu sichern. Der Code liegt dabei im Internet für jeden, mit den richtigen Zugangsdaten, erhältlich.}}
}

\newglossaryentry{ack}{
	name={Acknowledgement-Packet},
	description={\enquote{Das Acknowledgement-Packet oder Antwort-Paket und wird bei  Datenübertragungen verwendet, um den Erhalt von Daten zu bestätigen.}}
}

\newglossaryentry{speicherabbild}{
	name={Speicherabbild},
	description={\enquote{Ist eine Abbildung eines Datenträgers oder Datenspeichers, welche in einer Datei gespeichert werden kann}}
}

\newglossaryentry{zps}{
	name={ZPS},
	description={\enquote{Die Abkürzung für den Zwischenplattform-Server, welcher zwischen den Kartenlesern und der \gls{lba} steht}}
}

