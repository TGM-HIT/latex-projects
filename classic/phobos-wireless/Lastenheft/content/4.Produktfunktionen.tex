\section{Produktfunktionen}
Das Produkt wird in Desktop und App-Version unterteilt. Zusätzliche kommt noch die Publizierung des Produktes dazu.

\input{content/4.1.Desktop}
\input{content/4.2.App}

\subsection{Publizierung}
Um mit dem Produkt etwas zu erwirtschaften, muss dieses der Zielgruppe anschaulich gemacht werden. Dabei ist die größe der Plattformen und die Anzahl der Plattformen, welche dies machen, für ein erfolgreiches Produkt besonders wichtig. 
\begin{indentE}\mbox{}
	\paragraph{/LF3110/ Website erstellen}\mbox{}\\
	Es wird eine Webseite erstellt, auf welche das Produkt vorgestellt wird. Außerdem soll diese dem Besucher das Projektteam und den Auftraggeber näher bringen, auf einer eigenen About-Seite. Diese Webseite wird dann auf einen Hosting-Provider hochgeladen, von wo sie mit einer Domain aufgerufen werden kann.
	\paragraph{/LF3120/ Produkt auf Webseite veröffentlichen}\mbox{}\\
	Das Produkt muss auf ihrer eigenen Webseite /LF2060/ publiziert werden. Diese besitzt eine eigne Seite für den Download der Software. Dort kann man die App und Desktop Version für 5€ kaufen und die Desktop Version einzeln für 3€.
	\paragraph{/LF3210/ Produkt auf Play Store veröffentlichen}\mbox{}\\
	Die Android App soll auf dem Google Play Store hochgeladen werden und dort für eine Preis von 3€ verkauft werden. Dafür muss ein Google Developer Account erstellt werden, um diese hochzuladen.
	\paragraph{/LF3220/ Produkt auf App Store veröffentlichen}\mbox{}\\
	Die iOS App kann auf dem Apple App Store hochgeladen werden und dort für 3€ verkauft werden.
\end{indentE}