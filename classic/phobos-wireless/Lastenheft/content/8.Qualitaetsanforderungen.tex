\section{Qualitätsanforderungen}
Es werden die Wichtigkeit der verschiedenen Produktqualitäten festgehalten. Diese sollen vom Auftragnehmer bei der Priorisierung in acht genommen werden. 
Außerdem wird angemerkt das auf die Benutzbarkeit des Produkts besonders großen Wert gelegt werden soll, darunter fallen das Benutzerinterface mit welche der Benutzer mit dem Produkt interagiert. Dieses soll Fehlerfrei funktionieren und dem Benutzer mitteilen was momentan im Hintergrund passiert. Das nächst wichtigste ist die Zuverlässigkeit, diese befasst die Sicherheit der Daten, während dem Datentransfer, als auch der erfolgreiche Transfer.
\\\\
Unter \textbf{Funktionalität} wird das problemlose Funktionieren des Produktes beschrieben. Diese ist zwar nicht die wichtigste Qualität, bei Problemen sollte der Benutzer jedoch verständigt werden.
\\\\
Die \textbf{Zuverlässigkeit} bedeutet im Kontext des Produktes die sichere Übertragung. Auf diese soll besonders Wert gelegt werden da Dritte auf diese keinen Zugriff haben dürfen.
\\\\
Mit der \textbf{Benutzbarkeit} ist die Verwendung des Produkts gemeint. Diese soll, für den Benutzer, so simpel wie Möglich sein weswegen darauf hohen Wert gelegt wird.
\\\\
Die \textbf{Effizienz} steht in Verbindung mit der Geschwindigkeit der Datenübertragung. Auf welche zwar Wert gelegt werden soll aber nicht höchster Priorität entspricht.
\\\\
Unter \textbf{Änderbarkeit} wird die zukünftige Veränderbarkeit des Projektes für Dritte gezählt, dies kann durch zusätzliche Dokumentation oder einheitlicher und Programmierung erzielt werden. Dies ist aber für diese Projekt irrelevant.
\\\\
Die \textbf{Übertragbarkeit} bezeichnet die Möglichkeit das Endprodukt auf verschiedene Betriebssystemen zu benutzen. Hierbei wird besonders Wert auf Windows und Android gelegt. 
\begin{table}[H]
	\begin{center}
		\begin{tabularx}{\linewidth}{|X|c|x|c|c|}
			\hline
			\textbf{Produktqualität} & \textbf{sehr gut} & \textbf{gut} & \textbf{normal} & \textbf{irrelevant}\\
			\hline
			\textbf{Funktionalität} & &\textbf{X} & &\\
			\hline
			\textbf{Zuverlässigkeit} & \textbf{X}& & &\\
			\hline
			\textbf{Benutzbarkeit} &\textbf{X} & & &\\
			\hline
			\textbf{Effizienz} & & & \textbf{X} &\\
			\hline
			\textbf{Änderbarkeit} & & & &\textbf{X}\\
			\hline
			\textbf{Übertragbarkeit} & & \textbf{X} &  &\\
			\hline
		\end{tabularx}
	\end{center}
\end{table}