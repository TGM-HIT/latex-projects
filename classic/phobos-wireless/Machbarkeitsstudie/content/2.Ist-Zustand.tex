\section{Ist-Zustand}
Zu dem jetzigen Zeitpunkt ist die Nachfrage nach Datenübertragungs-Möglichkeiten definitiv da. Die Auswahl an Software, welche Cross Plattform Übertragung unterstützt, ist derzeit jedoch noch relativ gering. Es gibt zwar schon einige Möglichkeiten Daten zu übertragen, die meisten sind jedoch nur für ein bestimmtes Betriebssystem gedacht. Ebenfalls brauchen nur die wenigsten Applikationen keine stetige Internetverbindung.
\subsection*{Konkurrenzanalyse}
Weil es zu dem jetzigen Zeitpunkt noch keine Applikation gibt, welche die angeforderten Funktionen besitzt, ist dieses Projekt einzigartig. Es gibt zwar schon einige ähnliche Applikationen wie AirDrop für Apple Geräte und Zapya für Windows, diese erfüllen jedoch nicht die  Kernfunktionen welche das Produkt so einzigartig macht.\\\\
Hierbei erkennt AirDrop nur gewisse Apple Geräte welche in der nahen Umgebung des Benutzers sind  und hat dadurch keine Cross-Plattform Datenübertragung. Zu den verfügbaren Produkten gehört das iPhone, iPad, iPod Touch als auch jedes Mac Book und iMac mit dem Betriebssystem OS X 10.10 (Yosemite) oder neuer. Um eine erfolgreiche Datenübertragung mit AirDrop zu gewährleisten wird eine stetige Bluetooth Verbindung zwischen den Geräten als auch eine Internetverbindung benötigt.\\\\
Bei Zapya hingegen ist die Cross-Plattform Übertragung zwischen Windows, Android, iOS und Mac OS X möglich. Ebenfalls wird eine stetige Bluetooth Verbindung zwischen den benutzen Geräten benötigt. Im Gegensatz zu AirDrop muss für die Verwendung von Zapya jeoch keine Internetverbindung vorhanden sein. 