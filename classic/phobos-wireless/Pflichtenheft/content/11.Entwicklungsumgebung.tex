\section{Entwicklungsumgebung}
Die Entwicklungsumgebung beschreibt die Umstände der Software und Hardware, sowie der Orgware des Projektteams.
\subsection{Software}
Das Finalprodukt wird über eine Auswahl an vorinstallierten Anwendungen und integrierten Entwickleroberflächen auf den Rechnern des Projektteams geschaffen. Es werden keine weiteren Software-Lizenzen benötigt, da diese bereits gegeben sind. Dies senkt einmalige Zusatzkosten auf das Minimum. Zum Einsatz kommt hierbei die IDE Eclipse von Oracle mit dem internen WindowBuilder-Plugin.
\subsection{Hardware}
Um eine auseinandersetzungsfreie Entwicklung zu ermöglichen, werden private Notebooks und Rechensysteme für das Entwickeln des Endprodukts des Projektteams verwendet. Die Benutzung der eigenen Hardware schließt aus, dass Funktionalitäten der Software nicht auf allen System ordnungsgemäß betrieben werden können. Die Benutzung der Software setzt Betriebssysteme wie Linux-Distributionen und Windows 10, sowie die neuste Version von MacOS voraus.
\subsection{Entwicklungsschnittstellen}
Für die Datenübertragung werden gleich der späteren Anwendungen BlueTooth-Schnittstellen benutzt, da das System sonst nicht korrekt getestet und zum Laufen gebracht werden kann. Für die Orgware ist eine Internetschnittstelle von Nöten, da die meisten Zusatzmaterialien online abgesichert gespeichert sind.