\section{Vertragsgegenstand}
Es wird der Lieferumfang als auch die Produktbezogenen Leistungen ,welche vom Auftraggeber nach Übergabe der Produkte angefordert werden können, beschrieben.
\subsection{Lieferumfang}
Nach Abschluss des Projekts wird dem Auftraggeber eine Funktionsfähige Desktop-Applikation wie auch eine Smartphone-Version und ein Webportal übergeben. All diese Produkte enthalten, alle in dem Pflichtenheft enthaltenen, Produktfunktionen. Zu Installation für die beiden Applikationen, wird es eine Installationsdatei geben welche sich von der passenden Plattform heruntergeladen werden kann. Nach download der Datei kann diese ausgeführt werden und der Benutzer kann die Applikation erfolgreich installieren. 
\\\\
Mit der Lieferung der Produkte, werden jegliche Rechte an den Auftraggeber übergeben.
\subsection{Produktbezogene Leistungen}
Bis ein Monat nach Produktübergabe ist der Auftragnehmer verpflichtet jegliche Fehler in der Software zu beheben. Nach dieser Frist kann sich der Auftraggeber bei Bedarf wieder an den Auftragnehmer wenden und eine Fehlerbehebung anfordern. Je nach Fehlergröße muss der Auftraggeber natürlich einen entsprechenden Preis bezahlen. Ebenfalls kann man auch Patches für die Software anfordern. Der Preis wird, genau wie die Fehlerbehebung, nach dem Aufwand bezahlt.