\section{Produkteinsatz}
Es werden die Zielgruppen als auch der Anwendungsbereich, sowie die Bedienung zu den Verschiedenen Produkten des Projekts erläutert.
\subsection{Desktop Applikation}
Zielgruppe: Käufer des Produkts
\\
Anwendungsbereich: Datentransfer zwischen zwei Geräten
\\
Betriebsbedienung:
\\
Wie bereits im Lastenheft und in der Machbarkeitsstudie erklärt wird die Desktop Applikation für Laptops oder Heimcomputer entwickelt. Die Benutzung des Produkts ist so einfach wie möglich gestaltet weswegen man mit nur wenigen Knopfdrücken die zu versendende Datei als auch den Empfänger auswählen kann. Bevor der Benutzer jedoch die Daten endgültig absenden kann muss dieser noch einen Verschlüsselungskey eingeben. Nachdem dieser eingegeben wurde muss nur noch ein Bestätigungs-Knopf gedrückt werden mit dem die ausgewählte Datei an das zuvor ausgewählte Gerät gesendet wird.
\\\\
Erläuterung:
\\
Die, oben genannte, Zielgruppe benötigt das Produkt um in unerwarteten und zeit begrenzten Situation Daten zu übertragen. Um die Zeit welche, zum auswählen von Daten und Empfänger, so kurz wie möglich zu halten wird ein simples GUI verwendet. Um die Applikation so Effizient wie möglich zu benutzen werden jegliche Funktionen vor dem Benutzer versteckt und im Hintergrund ausgeführt. Nur Nach Beendung der Funktion wird der Kunde benachrichtigt.
\subsection{Smartphone Applikation}
Zielgruppe: Käufer des Produkts
\\
Anwendungsbereich: Datentransfer zwischen zwei Geräten
\\
Betriebsbedienung:
\\
Siehe Desktop Applikationen.
\\\\
Erläuterung:
\\
Siehe Desktop Applikation.
\subsection{Webportal}
Zielgruppe: Interessenten des Produkts
\\
Anwendungsbereich: Informationsausgabe über das Produkt
\\
Betriebsbedienung:
\\
Die Webseite wird auf den Browsern Google Chrome und Opera optimal angezeigt werden. Ebenfalls kann der Besucher das Webportal auf einem Mobilen Gerät wie einem Smartphone oder Tablet als auch auf einem Laptop oder Heimcomputer besuchen. Um die Inhalte welche sich auf der Seite befinden wird kein Benutzeraccount benötigt. Zum freien navigieren auf der Webseite wird es ein Globales Navigationssystem geben.
\\\\
Erläuterung:
\\
Da die Browser, Google Chrome und Opera, eine der meistbenutzten sind wird die Webseite speziell für diese beiden entwickelt. 