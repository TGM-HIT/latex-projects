\section{Qualitätsbestimmung}
Es wird das Produkt anhand der zu erbringenden Qualitäten beschrieben. Diese Qualitäten werden anhand dessen Auswirkung auf das Projekt einzeln beschreibend

\begin{table}[H]
	\begin{center}
		\begin{tabularx}{\linewidth}{|X|c|c|c|c|}
			\hline
			\textbf{Produktqualität}&Sehr gut&Gut&Normal&Nicht relevant\\
			\hline
			Funktionalität&X&&&\\
			\hline
			Angemessenheit&&&X&\\
			\hline
			Richtigkeit&X&&&\\
			\hline
			Interoperabilität&X&&&\\
			\hline
			Ordnungsmäßigkeit&X&&&\\
			\hline
			Sicherheit&&X&&\\
			\hline
			Zuverlässigkeit&&X&&\\
			\hline
			Reife&&&&X\\
			\hline
			Fehlertoleranz&&X&&\\
			\hline
			Wiederherstellbarkeit&&&X&\\
			\hline
			Benutzbarkeit&X&&&\\
			\hline
			Verständlichkeit&X&&&\\
			\hline
			Erlernbarkeit&&X&&\\
			\hline
			Bedienbarkeit&&X&&\\
			\hline
			Effizienz&&&X&\\
			\hline
			Zeitverhalten&&&X&\\
			\hline
			Verbrauchsverhalten&&&&X\\
			\hline
			Änderbarkeit&&X&&\\
			\hline
			Analysierbarkeit&&&&X\\
			\hline
			Modifizierbarkeit&&X&&\\
			\hline
			Stabilität&&&X&\\
			\hline
		\end{tabularx}
	\end{center}
\end{table}

\paragraph{Funktionalität}Es wird auf die Funktionen bezogenen und wie gut diese umgesetzt werden müssen
\paragraph{Angemessenheit}Es wird auf den Aufwand bezogenen und wie groß dieser sein soll zu Benutzung
\paragraph{Richtigkeit}Es wird auf die Richtigkeit(Komplettheit) der transferierten Daten bezogen
\paragraph{Interoperabilität}Es wird sich auf die Kompatibilität der Produktes mit verschiedenen Plattformen bezogen
\paragraph{Ordnungsgemäß}Es wird auf die Anzahl der Fehler bezogen, welche während der Sitzung aufkommen können
\paragraph{Sicherheit}Es wird sich auf die Sicherheit des Datentransfers bezogen
\paragraph{Zuverlässigkeit}Es wird sich auf die Zuverlässig Datenübertragung (keine Abbrüche) bezogen
\paragraph{Reife}Es wird sich auf die Qualität des Projektcodes bezogen
\paragraph{Fehlertoleranz}Es wird sich auf die Fähigkeit bezogen Fehler zu erkennen und richtig zu verwerten (kein Programmabsturz)
\paragraph{Wiederherstellbarkeit}Es wird sich auf die Wiederherstellbarkeit der Daten bei fehlerhaften Datentransfer bezogen
\paragraph{Benutzbarkeit}Es wird sich auf die Fähigkeit bezogen die Anwendung mit so wenigen Aktionen wie möglich zu bedienen
\paragraph{Verständlichkeit}Es werden sich auf Hilfsnachrichten bezogen, welche dem Benutzer bei der Verwendung helfen
\paragraph{Erlernbarkeit}Es wird sich auf die Möglichkeit bezogene das Programm ohne lesen der Hilfsnachrichten zu benutzen
\paragraph{Bedienbarkeit}Es wird sich auf die Bedienbarkeit aufgrund der benutzten Sprache im Programm bezogen
\paragraph{Effizienz}Es wird sich auf die Effizienz der vom Programm vollbrachten Funktionen bezogen anhand Ausführungszeit
\paragraph{Zeitverhalten}Es wird sich auf die Möglichkeit bezogen das Programm in naher Zukunft weiter zu benutzen
\paragraph{Verbrauchsverhalten}Es wird sich auf die vom Programm benötigten Ressourcen bezogen
\paragraph{Änderbarkeit}Es wird sich auf die Personalisierbarkeit des Programmes and den individuellen Benutzer bezogen
\paragraph{Stabilität}Es wird sich auf die Abstürze des Programms bezogen
