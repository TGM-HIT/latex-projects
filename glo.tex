\newacronym{ac-ci}{CI}{Continuous Integration}
\newglossaryentry{gls-ci}{
	name={Continuous Integration},
	description={''Kontinuierliche Integration (auch fortlaufende oder permanente Integration; englisch continuous integration) ist ein Begriff aus der Software-Entwicklung, der den Prozess des fortlaufenden Zusammenfügens von Komponenten zu einer Anwendung beschreibt.'' \cite{wiki-ci}}
}

\newacronym{ac-cd}{CD}{Continuous Delivery}
\newglossaryentry{gls-cd}{
	name={Continuous Delivery},
	description={''Continuous Delivery (CD) bezeichnet eine Sammlung von Techniken, Prozessen und Werkzeugen, die den Softwareauslieferungsprozess (englisch: Deployment) verbessern.'' \cite{wiki-cd}}
}

\newacronym{ac-webgl}{WebGL}{Web Graphics Library}
\newglossaryentry{gls-webgl}{
	name={WebGL},
	description={''WebGL (Web Graphics Library) ist eine JavaScript Programmierschnittstelle, mit deren Hilfe 3D-Grafiken hardwarebeschleunigt im Webbrowser ohne zusätzliche Erweiterungen dargestellt werden können.'' \cite{wiki-webgl}}
}

\newacronym{ac-ape}{APE}{Ajax Push Engine}
\newacronym{ac-cid}{CId}{Corporate Identity}
\newacronym{ac-js}{JS}{JavaScript}
\newacronym{ac-css}{CSS}{Cascading Stylesheets}
\newacronym{ac-npm}{npm}{Node Package Manager}
\newacronym{ac-ui}{UI}{User Interface}
\newacronym{ac-cfo}{CFO}{Chief Financial Officer}
\newacronym{ac-ceo}{CEO}{Chief Executive Officer}
